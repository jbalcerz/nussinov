%%%%%%%%%%%%%%%%%%%%%%%%%%%%%%%%%%%%%%%%%
% Short Sectioned Assignment
% LaTeX Template
% Version 1.0 (5/5/12)
%
% This template has been downloaded from:
% http://www.LaTeXTemplates.com
%
% Original author:
% Frits Wenneker (http://www.howtotex.com)
%
% License:
% CC BY-NC-SA 3.0 (http://creativecommons.org/licenses/by-nc-sa/3.0/)
%
%%%%%%%%%%%%%%%%%%%%%%%%%%%%%%%%%%%%%%%%%

% TeX encoding = utf8
% TeX spellcheck = pl_PL

%----------------------------------------------------------------------------------------
%	PACKAGES AND OTHER DOCUMENT CONFIGURATIONS
%----------------------------------------------------------------------------------------

\documentclass[paper=a4, fontsize=11pt]{scrartcl} % A4 paper and 11pt font size

%_________________________________________________________________
%dopisane przeze mnie:

\usepackage[utf8]{inputenc}
\usepackage{polski}
\sloppy %zakaz wydłużania lini (gdzy nie może złożyć)
\clubpenalty=10000 %to kara za sierotki 
\widowpenalty=10000 %nie pozostawia wdów 
\brokenpenalty=10000 %nie dieli wyrazów pomiędzy stronami 
\usepackage{indentfirst} %wcina 1 akapit (wg polskiej typografii wskazane) 
\usepackage{color,graphicx}
\usepackage{enumerate}
\usepackage{url}
%_________________________________________________________________

\usepackage[T1]{fontenc} % Use 8-bit encoding that has 256 glyphs
\usepackage{fourier} % Use the Adobe Utopia font for the document - comment this line to return to the LaTeX default
\usepackage[english]{babel} % English language/hyphenation
\usepackage{amsmath,amsfonts,amsthm} % Math packages

\usepackage{lipsum} % Used for inserting dummy 'Lorem ipsum' text into the template

\usepackage{sectsty} % Allows customizing section commands
\allsectionsfont{\centering \normalfont\scshape} % Make all sections centered, the default font and small caps

\usepackage{fancyhdr} % Custom headers and footers
\pagestyle{fancyplain} % Makes all pages in the document conform to the custom headers and footers
\fancyhead{} % No page header - if you want one, create it in the same way as the footers below
\fancyfoot[L]{} % Empty left footer
\fancyfoot[C]{} % Empty center footer
\fancyfoot[R]{\thepage} % Page numbering for right footer
\renewcommand{\headrulewidth}{0pt} % Remove header underlines
\renewcommand{\footrulewidth}{0pt} % Remove footer underlines
\setlength{\headheight}{13.6pt} % Customize the height of the header

\numberwithin{equation}{section} % Number equations within sections (i.e. 1.1, 1.2, 2.1, 2.2 instead of 1, 2, 3, 4)
\numberwithin{figure}{section} % Number figures within sections (i.e. 1.1, 1.2, 2.1, 2.2 instead of 1, 2, 3, 4)
\numberwithin{table}{section} % Number tables within sections (i.e. 1.1, 1.2, 2.1, 2.2 instead of 1, 2, 3, 4)

\setlength\parindent{0pt} % Removes all indentation from paragraphs - comment this line for an assignment with lots of text

%----------------------------------------------------------------------------------------
%	TITLE SECTION
%----------------------------------------------------------------------------------------

\newcommand{\horrule}[1]{\rule{\linewidth}{#1}} % Create horizontal rule command with 1 argument of height

\title{	
\normalfont \normalsize 
\textsc{Metody bioinformatyki: Dokumentacja wstępna projektu} \\ [25pt] % Your university, school and/or department name(s)
\horrule{0.5pt} \\[0.4cm] % Thin top horizontal rule
\huge Implementacja algorytmu Nussinov do obliczania struktury drugorzędowej DNA i RNA \\ % The assignment title
\horrule{2pt} \\[0.5cm] % Thick bottom horizontal rule
}

\author{Jerzy Balcerzak,
Oleg Druzhynets} % Your name

\date{\normalsize\today} % Today's date or a custom date

\begin{document}

\maketitle % Print the title

%----------------------------------------------------------------------------------------
%	PROBLEM 1
%----------------------------------------------------------------------------------------

Jednym z podstawowych problemów nad którego rozwiązaniem pracuje wiele naukowców badających kwasy rybonukleinowe (RNA) to określenie trójwymiarowej struktury molekuł. Jednak w przypadku kwasów RNA, ostateczna budowa molekuły jest w dużej mierze zdeterminowana poprzez drugorzędową strukturę RNA. Na przełomie lat 70' i 80' ubiegłego wieku, zespół badawczy dr Ruth Nussinov zaproponował metodykę przewidywania możliwych drugorzędowych struktur jednoniociowych kwasów RNA \cite{bib:nuss1}\cite{bib:nuss2} wykorzystując w niej programowanie dynamiczne – to właśnie implementacji tego algorytmu będzie dotyczył poniżej przedstawiony projekt.

\section{Cel projektu}

	Celem projektu jest implementacja programu obliczającego strukturę drugorzędową cząsteczki DNA lub RNA wykorzystując algorytm Nussinov.

%----------------------------------------------------------------------------------------

\section{Skład zespołu}

	Jerzy Balcerzak \\
	Oleg Druzhynets 
%------------------------------------------------

\section{Funkcjonalność aplikacji}

Tworzony w ramach projektu program będzie miał naturę aplikacji konsolowej - po jego uruchomieniu, użytkownik będzie mógł wprowadzić ciągi znaków wczytując wskazany plik, bądź też wpisując dane z klawiatury. 
Program zwróci czytelną reprezentację struktury drugorzędowej na konsolę oraz również do pliku.

\section{Przewidziane prace}

Jako źródło przykładowych danych testowych, wraz z wynikami referencyjnymi zostanie wykorzystana baza danych RNA STRAND v2.0 \cite{bib:baza}. Zatem aplikacja zostanie dostarczona wraz ze zbiorem przykładowych zestawów danych które użytkownik będzie mógł uruchomić i zapoznać się z działaniem aplikacji oraz zwracanym formatem danych.\\

Ponadto zostaną przeprowadzone testy wydajnościowe...?
%----------------------------------------------------------------------------------------

\renewcommand{\refname}{\normalfont\selectfont\normalsize Literatura i źródła} 


\begin{thebibliography}{1}

\bibitem{bib:nuss1}
Nussinov, Ruth; Pieczenik, George; Griggs, Jerrold R.; Kleitman, Daniel J. (1 July 1978). "Algorithms for Loop Matchings". SIAM Journal on Applied Mathematics 35 (1): 68–82. doi:10.1137/0135006.

\bibitem{bib:nuss2}
Nussinov, R; Jacobson, AB (Nov 1980). "Fast algorithm for predicting the secondary structure of single-stranded RNA.". Proceedings of the National Academy of Sciences of the United States of America 77 (11): 6309–6313. doi:10.1073/pnas.77.11.6309. PMC 350273. PMID 6161375.
\bibitem{bib:baza}
RNA STRAND v2.0 - The RNA secondary STRucture and statistical ANalysis Database,
\url{http://www.rnasoft.ca/strand/}


\end{thebibliography}
\end{document}