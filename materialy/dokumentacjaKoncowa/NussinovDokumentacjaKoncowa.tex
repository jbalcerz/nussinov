%%%%%%%%%%%%%%%%%%%%%%%%%%%%%%%%%%%%%%%%%
% Short Sectioned Assignment
% LaTeX Template
% Version 1.0 (5/5/12)
%
% This template has been downloaded from:
% http://www.LaTeXTemplates.com
%
% Original author:
% Frits Wenneker (http://www.howtotex.com)
%
% License:
% CC BY-NC-SA 3.0 (http://creativecommons.org/licenses/by-nc-sa/3.0/)
%
%%%%%%%%%%%%%%%%%%%%%%%%%%%%%%%%%%%%%%%%%

% TeX encoding = utf8
% TeX spellcheck = pl_PL

%----------------------------------------------------------------------------------------
%	PACKAGES AND OTHER DOCUMENT CONFIGURATIONS
%----------------------------------------------------------------------------------------

\documentclass[paper=a4, fontsize=11pt]{scrartcl} % A4 paper and 11pt font size

%_________________________________________________________________
%dopisane przeze mnie:

\usepackage[utf8]{inputenc}
\usepackage{polski}
\sloppy %zakaz wydłużania lini (gdzy nie może złożyć)
\clubpenalty=10000 %to kara za sierotki 
\widowpenalty=10000 %nie pozostawia wdów 
\brokenpenalty=10000 %nie dieli wyrazów pomiędzy stronami 
\usepackage{indentfirst} %wcina 1 akapit (wg polskiej typografii wskazane) 
\usepackage{color,graphicx}
\usepackage{enumerate}
\usepackage{url}
%_________________________________________________________________

\usepackage[T1]{fontenc} % Use 8-bit encoding that has 256 glyphs
\usepackage{fourier} % Use the Adobe Utopia font for the document - comment this line to return to the LaTeX default
\usepackage[english]{babel} % English language/hyphenation
\usepackage{amsmath,amsfonts,amsthm} % Math packages

\usepackage{lipsum} % Used for inserting dummy 'Lorem ipsum' text into the template

\usepackage{sectsty} % Allows customizing section commands
\allsectionsfont{\centering \normalfont\scshape} % Make all sections centered, the default font and small caps

\usepackage{fancyhdr} % Custom headers and footers
\pagestyle{fancyplain} % Makes all pages in the document conform to the custom headers and footers
\fancyhead{} % No page header - if you want one, create it in the same way as the footers below
\fancyfoot[L]{} % Empty left footer
\fancyfoot[C]{} % Empty center footer
\fancyfoot[R]{\thepage} % Page numbering for right footer
\renewcommand{\headrulewidth}{0pt} % Remove header underlines
\renewcommand{\footrulewidth}{0pt} % Remove footer underlines
\setlength{\headheight}{13.6pt} % Customize the height of the header

\numberwithin{equation}{section} % Number equations within sections (i.e. 1.1, 1.2, 2.1, 2.2 instead of 1, 2, 3, 4)
\numberwithin{figure}{section} % Number figures within sections (i.e. 1.1, 1.2, 2.1, 2.2 instead of 1, 2, 3, 4)
\numberwithin{table}{section} % Number tables within sections (i.e. 1.1, 1.2, 2.1, 2.2 instead of 1, 2, 3, 4)

\setlength\parindent{0pt} % Removes all indentation from paragraphs - comment this line for an assignment with lots of text

%----------------------------------------------------------------------------------------
%	TITLE SECTION
%----------------------------------------------------------------------------------------

\newcommand{\horrule}[1]{\rule{\linewidth}{#1}} % Create horizontal rule command with 1 argument of height

\title{	
\normalfont \normalsize 
\textsc{Metody bioinformatyki: Dokumentacja końcowa projektu} \\ [25pt] % Your university, school and/or department name(s)
\horrule{0.5pt} \\[0.4cm] % Thin top horizontal rule
\huge Implementacja algorytmu Nussinov do obliczania struktury drugorzędowej DNA i RNA \\ % The assignment title
\horrule{2pt} \\[0.5cm] % Thick bottom horizontal rule
}

\author{Jerzy Balcerzak,
Oleg Druzhynets} % Your name


\date{\normalsize\today} % Today's date or a custom date

\begin{document}

\maketitle % Print the title

%----------------------------------------------------------------------------------------
%	PROBLEM 1
%----------------------------------------------------------------------------------------

\section{Cel projektu}

	Celem projektu była implementacja programu obliczającego strukturę drugorzędową cząsteczki DNA lub RNA wykorzystując algorytm Nussinov.

%----------------------------------------------------------------------------------------


%------------------------------------------------


\section{Użyte technologie}

Stworzony program został całkowicie przygotowany w języku Python w standardowej wersji 2.7. Wybór padł właśnie na ten język programowania przede wszystkim ze względu na jego obiektowy charakter, przejrzystość składni oraz bogactwo bibliotek. 

Duża część kodu głównego algorytmu oraz modułu obsługi wejścia została przygotowywana w duchu metodyki Test Driven Developement (TDD). Świetnym narzędziem wykorzystanym podczas prac okazał się więc framework pyUnit.

Ponaddto, do testów gotowego już programu wykorzystano: ... Wizualizacja otrzymanych wyników została oparta o bibliotekę networkx.

\section{Osiągnięta funkcjonalność aplikacji}

W ramach projektu stworzono prostą w obsłudze aplikację konsolową (nussinovCalculator.py). Jej funkcjonalność można opisać następująco:

\begin{itemize}
	\item Pomoc dla użytkownika - wywołanie programu z parametrem -h wyświetla pulę obsługiwanych opcji oraz ich domyślne parametry.
	\item Wejście - wywołanie programu z opcją -i oraz podaną nazwą pliku (np. ./nussinovCalculator.py -i "NazwaPliku.txt") spowoduje odczytanie z pliku ciągów nukleotydów (można podać ich wiele w kilku linijkach) i przeprowadzenie algorytmu nussinov dla każdego z nich. Program można wywołać bez tej opcji (wtedy domyślnie odczytywane są dane z pliku "input.txt"), jak również podać ciąg nukleotydów prosto do konsoli używając opcji \mbox{-r} (np. \mbox{-r} GGGAAAACCC)
	\item Wyjście - domyślnie program wypisuje wynik na konsolę oraz zwraca plik "output.txt" w postaci listy par indeksów nukleotydów które będą połączone w strukturze drugorzędowej. Uzywając opcji -o można wskazać plik do którego dane mają być zapisane (np. -o outputs/out1.txt).
	\item Parametryzacja algorytmu - przy uruchamianiu programu użytkownik może dodatkowo wprowadzić:
		\begin{itemize}
			\item minimalną ilość nukleotydów w pętli używając opcji \mbox{-m} (domyślnie: \mbox{-m} 4),
			\item oraz wartości w macierzy energii (opcja -e) w formacie listy list (np. \mbox{-e} [[0,0,0,2],[0,0,3,0],[0,3,0,3],[2,0,1,0]] ). Przy czym kolejne kolumny i wiersze odnoszą się do nazw nukleotydów w kolejności alfabetycznej - A, C, G, U.
		\end{itemize}
	
	\item Wizualizacja
	
	
	
	



\end{itemize}



\section{Opis zastosowanego algorytmu}

Prezentowany program rozwiązuje następujące zadanie: "Określ strukturę drugorzędową dla danej sekwencji nukleotydów". Do rozwiązania tego problemu użyto algorytmu zaproponowanego przez profesor Nussinov \cite{bib:nuss1}\cite{bib:nuss2} opierającego swoje działanie na metodyce programowania dynamicznego. Ogólnikowo, przebieg algorytmu przedstawiają się następująco:

\begin{itemize}

	\item Przygotowanie macierzy Nussinov oraz uzupełnienie jej wartości - implementacja tej części w głównej mierze została zainspirowana informacjami przekazanymi w wykładzie dr Roberta Nowaka \cite{bib:slajdyNowak})
	\item Przejście zwrotne po macierzy w celu znalezienia par nukleotydów, tzw. Traceback. Istnieją różne możliwości implementacji tej procedury - w projekcie została wykorzystana jej elegancja wersja rekurencyjna (przedstawiona w materiałach udostępnionych przez uniwersytet w Tübingen \cite{bib:slajdyTubigen})
	
\end{itemize}


\section{Przeprowadzone testy}

\subsection{Testy jednostkowe}
o testach jednostkowych 

\subsection{Zrealizowane testy}
Zrealizowane zostały 4 tak zwane testy “czarnej skrzynki”: 2 testy czasu wykonania programu, profilowanie kodu linijka po linijce a także profilowanie pamięci wykorzystywanej programem:
1. timeTest.py - nasza wersja testowania czasu wykonania, korzysta z pomocniczej klasy Timer;
2. timetest.sh - oszacowanie czasu wykonania programu z wykorzystaniem standardowej UNIX komendy time;
Inne dwa testujące skrypty są bardziej skomplikowane. Szczegóły pracy z tymi skryptami są zamieszczone w poszczególnych skryptach:
3. line_profiler_script.sh - profilowanie kodu linijka po linijce;
4. memory_profiler_script.sh - profilowanie pamięci wykorzystywanej programem.

\subsection{Profiler linijka po linijce}
Dla testowania wydajności kodu skorzystaliśmy się z profilera Roberta Kerna (https://pythonhosted.org/line_profiler/). Za pomocą tego narzędzia można zobaczyć jak szybko i jak często wykonywana jest każda linijka kodu w naszym skrypcie. Żeby móc korzystać z line_profiler, najpierw musimy go zainstalować: 
               $ pip install line_profiler
Zainstalowane zostaną moduł "line_profiler" oraz skrypt "kernprof.py". Dla testowania kodu korzystamy z dekoratora "@profile", którego ustawiamy przed interesującymi nas funkcjami. Nic więcej nie potrzebno importować, ponieważ skrypt kernprof.py samodzielnie wkleja potrzebny kod w trakcie uruchomienia naszego skryptu. 
Po uruchomieniu skryptu line_profiler_script.sh odbywa się testowanie interesujących nas funkcji. Wyniki testowania zostają zapisane do pliku line_profiler_test.txt. 

\subsection{Profiler pamięci}
Na podstawie profilera Roberta Kerna przez Fabiana Pedregosę został stworzony profiler pamięci memory_profiler (https://github.com/fabianp/memory_profiler), który też cechuje się prostym i niezawodnym działanie. Korzystamy z danego profilera żeby przetestować ilość pamięci, wykorzystywanej przez zaimplementowany nami skrypt. Żeby móc korzystać z memory_profiler, musimy go zainstalować: 

                   $ pip install -U memory_profiler
                        $ pip install psutil
Instalujemy także psutil, co sprawi lepszą wydajność profilera. 
Dla testowania kodu również korzystamy z dekoratora "@profile", którego ustawiamy przed interesującymi nas funkcjami. Nic więcej nie potrzebno importować. Po uruchomieniu skryptu memory_profiler_script.sh odbywa się testowanie interesujących nas funkcji. Wyniki testowania zostają zapisane do pliku memory_profiler_test.txt. 

\subsection{Testy dla różnych sekwencji}

Poniżej przedstawione są wyniki działania skryptów testowych dla 2 różnych sekwencji: 

1. GGGAAAAAACCC

timeTest.py

=> elapsed time: 1.95107889175 s

timetest.sh

real	0m2.409s
user	0m0.503s
sys	0m0.112s

line_profiler_script.sh

File: /Users/sieteestrellas/Desktop/nussinov/src/nussinov.py
Function: _traceback at line 54
Total time: 7.1e-05 s

Line #      Hits         Time  Per Hit   % Time  Line Contents
==============================================================
    54                                               @profile
    55                                               def _traceback(self,i,j):
    56         9           12      1.3     16.9          if math.isnan(self._sMatrix[i][j]):
    57                                                       raise NussinovException(2,'sMatrix is not build yet!')
    58                                                       return
    59         9            6      0.7      8.5          if i < j:
    60         8            7      0.9      9.9              if self._sMatrix[i][j] == self._sMatrix[i+1][j]:
    61         5           18      3.6     25.4                  self._traceback(i+1,j)
    62         3            3      1.0      4.2              elif self._sMatrix[i][j] == self._sMatrix[i][j-1]:
    63                                                           self._traceback(i,j-1)
    64         3            3      1.0      4.2              elif self._sMatrix[i][j] == (self._sMatrix[i+1][j-1] 
    65         3           10      3.3     14.1                                           + self._getEnergy(self._chain[i],self._chain[j])):
    66         3            4      1.3      5.6                  self._outputPairs.append((i,j))
    67         3            8      2.7     11.3                  self._traceback(i+1,j-1)
    68                                                       else:
    69                                                           for k in range(i+1,j):
    70                                                               if self._sMatrix[i][j] == (self._sMatrix[i][k] 
    71                                                                                          + self._sMatrix[k+1][j]):
    72                                                                   self._traceback(i,k)
    73                                                                   self._traceback(k+1,j)
    74                                                                   break

memory_profiler_script.sh

Filename: /Users/sieteestrellas/Desktop/nussinov/src/nussinov.py

Line #    Mem usage    Increment   Line Contents
================================================
    54   36.863 MiB    0.000 MiB       @profile
    55                                 def _traceback(self,i,j):
    56   36.863 MiB    0.000 MiB           if math.isnan(self._sMatrix[i][j]):
    57                                         raise NussinovException(2,'sMatrix is not build yet!')
    58                                         return
    59   36.867 MiB    0.004 MiB           if i < j:
    60   36.863 MiB   -0.004 MiB               if self._sMatrix[i][j] == self._sMatrix[i+1][j]:
    61   36.867 MiB    0.004 MiB                   self._traceback(i+1,j)
    62   36.844 MiB   -0.023 MiB               elif self._sMatrix[i][j] == self._sMatrix[i][j-1]:
    63                                             self._traceback(i,j-1)
    64   36.844 MiB    0.000 MiB               elif self._sMatrix[i][j] == (self._sMatrix[i+1][j-1] 
    65   36.844 MiB    0.000 MiB                                            + self._getEnergy(self._chain[i],self._chain[j])):
    66   36.844 MiB    0.000 MiB                   self._outputPairs.append((i,j))
    67   36.867 MiB    0.023 MiB                   self._traceback(i+1,j-1)
    68                                         else:
    69                                             for k in range(i+1,j):
    70                                                 if self._sMatrix[i][j] == (self._sMatrix[i][k] 
    71                                                                            + self._sMatrix[k+1][j]):
    72                                                     self._traceback(i,k)
    73                                                     self._traceback(k+1,j)
    74                                                     break



2. GUGAACCUGGCGGCGUGCCUAAUACAUGCAAGUCGAACGAUGAAGUCCUAGCUUGCUAGGAUGGAUUAGUGGCGCACGGGUGAGUAAUGUAUAGCUAAUCUGCCCCAUAGAGAGGAACAACACUUAGAAAUGAGUGCUAAUACCUCAUACUCCAAUUAUACAUAAGUUUAAUUGGGAAAUGUAGCUCUUAAUAAUAUAUAUCAAAGAUAAUUUAUAAAUAAAAAGUUAUAAAUAUAACAAUAAGCUUUUUUGAAGCUUUAUAUUAAUAAAGCGAAAAAAAAGCAAAGCAGUUAGAUUUAAUAAAUUUUUAUAGCAUUUAAAAAUACAAAAGACUUAAUUUUUAAAUCUAAAUAUAAAUUAUUACUAAUAUUUUUAAUAGGCAGAAAUGUUAAGACAGGUGCUGCACGGCUGUCGUCAGCUCGUGUCGUGAGAUGUUGGGUUAAGUCCCGCAACGAGCGCAACCCACGUGUUUAGUUGCUAACAGUUAGGCUGAGCACUCUAAACAGACUGCCUUCGUAAGGAGGAGGAAGGUGUGGACGACGUCAAGUCAUCAUGGCCCUUAUGUCCGGGGCGACACACGUGCUACAAUGGCAUAUACAAUAAGACGCAAUAUCGCGAGAUGGAGCAAAUCUAUAAAAUAUGUCCCAGUUCGGAUUGGAGUCUGCAACUCGACUCCAUGAAGCCGGAAUCGCUAGUAAUCGUAGAUCAGCCAUGCUACGGUGAAUACGUUCCCGGGUCUUGUACUCACCGCCCGUCACACCAUGGGAGUUGAUUUCACUCGAAGCCCAAAUACCAAAUUGGUUAUGGUCCACAGUGGAAUCAGCGACUGGGGUGAAGUCGUAACAAGGUAACCGUAGGAGAAC

timeTest.py

=> elapsed time: 49.3690121174 s

timetest.sh

real	0m45.319s
user	0m42.562s
sys	0m0.235s

line_profiler_script.sh

File: /Users/sieteestrellas/Desktop/nussinov/src/nussinov.py
Function: _traceback at line 54
Total time: 0.017021 s

Line #      Hits         Time  Per Hit   % Time  Line Contents
==============================================================
    54                                               @profile
    55                                               def _traceback(self,i,j):
    56       559          614      1.1      3.6          if math.isnan(self._sMatrix[i][j]):
    57                                                       raise NussinovException(2,'sMatrix is not build yet!')
    58                                                       return
    59       559          416      0.7      2.4          if i < j:
    60       541          615      1.1      3.6              if self._sMatrix[i][j] == self._sMatrix[i+1][j]:
    61       122          321      2.6      1.9                  self._traceback(i+1,j)
    62       419          449      1.1      2.6              elif self._sMatrix[i][j] == self._sMatrix[i][j-1]:
    63        81          287      3.5      1.7                  self._traceback(i,j-1)
    64       338          382      1.1      2.2              elif self._sMatrix[i][j] == (self._sMatrix[i+1][j-1] 
    65       338         1238      3.7      7.3                                           + self._getEnergy(self._chain[i],self._chain[j])):
    66       321          347      1.1      2.0                  self._outputPairs.append((i,j))
    67       321         5710     17.8     33.5                  self._traceback(i+1,j-1)
    68                                                       else:
    69      2499         1798      0.7     10.6                  for k in range(i+1,j):
    70      2499         2287      0.9     13.4                      if self._sMatrix[i][j] == (self._sMatrix[i][k] 
    71      2499         2390      1.0     14.0                                                 + self._sMatrix[k+1][j]):
    72        17           55      3.2      0.3                          self._traceback(i,k)
    73        17           60      3.5      0.4                          self._traceback(k+1,j)
    74        17           52      3.1      0.3                          break


memory_profiler_script.sh

Filename: /Users/sieteestrellas/Desktop/nussinov/src/nussinov.py

Line #    Mem usage    Increment   Line Contents
================================================
    54   55.719 MiB    0.000 MiB       @profile
    55                                 def _traceback(self,i,j):
    56   55.719 MiB    0.000 MiB           if math.isnan(self._sMatrix[i][j]):
    57                                         raise NussinovException(2,'sMatrix is not build yet!')
    58                                         return
    59   55.719 MiB    0.000 MiB           if i < j:
    60   55.719 MiB    0.000 MiB               if self._sMatrix[i][j] == self._sMatrix[i+1][j]:
    61   55.719 MiB    0.000 MiB                   self._traceback(i+1,j)
    62   55.719 MiB    0.000 MiB               elif self._sMatrix[i][j] == self._sMatrix[i][j-1]:
    63   55.719 MiB    0.000 MiB                   self._traceback(i,j-1)
    64   55.719 MiB    0.000 MiB               elif self._sMatrix[i][j] == (self._sMatrix[i+1][j-1] 
    65   55.719 MiB    0.000 MiB                                            + self._getEnergy(self._chain[i],self._chain[j])):
    66   55.719 MiB    0.000 MiB                   self._outputPairs.append((i,j))
    67   55.719 MiB    0.000 MiB                   self._traceback(i+1,j-1)
    68                                         else:
    69   55.715 MiB   -0.004 MiB                   for k in range(i+1,j):
    70   55.715 MiB    0.000 MiB                       if self._sMatrix[i][j] == (self._sMatrix[i][k] 
    71   55.715 MiB    0.000 MiB                                                  + self._sMatrix[k+1][j]):
    72                                                     self._traceback(i,k)
    73                                                     self._traceback(k+1,j)
    74   55.719 MiB    0.004 MiB                           break


 

\section{Porównanie z wynikami innych aplikacji}


o wynikach uzyskanych w innych aplikacjach 

\section{Wizualizacja wyników}

\section{Wnioski}

Jakieś wnioski

\renewcommand{\refname}{\normalfont\selectfont\normalsize Literatura i źródła} 

\bibliographystyle{plain}
\bibliography{NussinovDokumentacjaKoncowa}


\end{document}